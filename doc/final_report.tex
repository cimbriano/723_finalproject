\documentclass[11pt, letterpaper, oneside]{article}
\usepackage{enumerate}
\usepackage{ calc }
\usepackage{ amssymb }

\begin{document}

\title{CMSC 723: Final Project \\ Automatically Distinguishing American and British English}
\author{Christopher Imbriano, David Wasser}

\maketitle

%Proposal Questions:
%* what problem are you tackling
%* why is it important/interesting
%* what is your basic approach
%* where will you get data
%* how will you evaluate success
%* what you hope to get out of this project

% 4 pages length

% explain why it works (when it does)
% why it doesn't work (when it doesn't).

% You should also reserve about a page of this writeup to talk about what _didn't_ work and what was harder than you expected it to be (there _will_ be something!).
\section{Abstract}

NOTE:  Much of this is placeholder from the progress report and uses future tense.  Change to past tense.

\section{Problem}
Voice control and command recognition is a popular new feature of smartphones. 
Apple's Siri and Google's Search App are two examples of systems that take voice commands form users and performs actions like searching the web, creating calendar events, or sending message.
Users had been reporting that these systems have been having trouble understanding them, especially those with British or Southern American accents. 
Improvements in accent understanding could help systems like these better recognize users with variations in their speech, as well as improve the localization of the voice the system produces.

\section{Introduction}


\section{Data Sources}
	\begin{enumerate}
		\item CMU Pronunciation Dictionary \\ http://www.speech.cs.cmu.edu/cgi-bin/cmudict
		
		"A machine-readable pronunciation dictionary for North American English that contains over 125,000 words and their transcriptions."
		
		\item Oxford English Dictionary \\ http://public.oed.com/subscriber-services/sru-service
		
		They have a subscriber service which could be used to compile a corpus of words and their British-English pronunciation.
		
		\item Speech Accent Archive \\ http://accent.gmu.edu
		
		"The speech accent archive uniformly presents a large set of speech samples from a variety of language backgrounds. 
		Native and non-native speakers of English read the same paragraph and are carefully transcribed. "
	\end{enumerate}

\section{ Method }
Using a corpus of words for which we have both the American and British phonetic transcription, we will build a Probabilistic Finite State Acceptor.
We will build a language model over phonemes since we will assume phonetic transcriptions are our input data.
Once we have these two language models, one for each dialect, we can feed it phonetic transcriptions and receive a probability of that transcription in each of British and American English.
The distinguisher will produce a judgment based on the more probable dialect.


\section{ Results }

	\subsection{ Evaluation Method }
	This paragraph is from the progress report - needs updating.
	
	The Speech Accent Archive, produced by researchers at George Mason University, it a large data set of speakers of various languages speaking a sentence as well as the phonetic transcription of their speech.
	Included in the set of speakers are individuals from USA and the UK.
	We will use these two subsets as ground truth. So given the transcribed speech of speakers in those sets, we expect that our distinguisher should correctly identify each speaker's native dialect.
	Evaluation will then be a percentage of speakers correctly identified.
	

	

\section{Complications}

\begin{itemize}
\item Character sets in general.
	SAA raw data used unicode decimals
\item Phonetic granularity
	Stress
	Diacritics
	List the character mappings and the excluded characters
\item Disparate data sources for testing and model building
	OED - "Proper" pronunciations
	SAA - Real human speech with lots of variability

\end{itemize}




%%%%%%%%%%%%%%%%%%%%%%%%%%%%%%%%%%%%%%%%%%%%%%%%%%%%%%%%%%%%%%%%%%%%%%%%

\begin{thebibliography}{9}
%	\bibitem{sads} Elaine Shi, Charalampos Papamanthou \emph{Streaming Authenticated Data Structures}, Date Unknown
\end{thebibliography}

\end{document}
